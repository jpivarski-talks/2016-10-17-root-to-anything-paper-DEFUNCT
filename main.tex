\documentclass{article}
\usepackage[utf8]{inputenc}
\usepackage{fullpage}
\usepackage{hyperref}
\usepackage{array}
\usepackage{xcolor}

\title{Liberating ROOT data through Apache Arrow}
\author{Jim Pivarski}
\date{\today}

\newcolumntype{P}[1]{>{\centering\arraybackslash}b{#1}}

\begin{document}
\maketitle

\section*{Motivation}

The ROOT I/O system was introduced as a way to standardize High Energy Physics (HEP) experiment data in a self-describing format. It was proposed\footnote{\url{https://root.cern.ch/save-data}} as a safe alternative to creating custom file formats for each experiment, and in this, it succeeded. However, in the past two decades, open source projects from the Big Data industry have taken up the same charge and provide this function with higher performance and portability. 

The ROOT file format is a ``meta-format'' in the same sense as XML, JSON, HDF5, and now Avro, Thrift, Protocol Buffers, Parquet, ORC, and Feather. All of these file formats are generic containers of structured data. XML and JSON are human-readable text and therefore inefficient for large stores of numerical data. Avro, Thrift, and Protocol Buffers are binary record formats developed by the Hadoop project, Facebook, and Google (respectively) to transfer streams of data between applications, often across networks and programming languages. Parquet, ORC, and Feather are the most similar to ROOT: they are columnar stores for accessing large batches of data, usually in databases.

All of the file formats mentioned above differ from ROOT in the amount of seeking that is required to interpret a file. 




47,407 $t\bar{t}$ events (4-vectors and generator-level particles) containing 55,560 muons. (``Error while instantiating TBranchElement'')

blockSize = 32 MB

pageSize = 1 MB

the one field was Muon.pt for all muons

single processor (had to be explicitly forced for Parquet, which uses all available CPU cores by default)

\renewcommand{\arraystretch}{1.2}

\noindent\begin{tabular}{p{2.4 cm} | P{1.2 cm} P{1.9 cm} P{1.2 cm} P{1.8 cm} P{1.9 cm} P{1.2 cm} P{1.8 cm}}
File type & file size (MB) & convert from ROOT (sec) & read all in C++ (sec) & read a field in C++ (sec) & convert from Avro (sec) & read all in Java (sec) & read a field in Java (sec)      \\\hline
ROOT none     & 399 &          110 &           47 &         11.6 &       \textcolor{gray}{didn't} &        \textcolor{gray}{can't} &        \textcolor{gray}{can't} \\
ROOT gzip 1   & 204 &          127 &           48 &         12.0 &          \textcolor{gray}{try} & \textcolor{gray}{$\downarrow$} & \textcolor{gray}{$\downarrow$} \\
ROOT gzip 2   & 208 &          132 &           47 &         11.4 & \textcolor{gray}{$\downarrow$} & \textcolor{gray}{$\downarrow$} & \textcolor{gray}{$\downarrow$} \\
ROOT gzip 9   & 202 &          265 &           48 &         11.2 & \textcolor{gray}{$\downarrow$} & \textcolor{gray}{$\downarrow$} & \textcolor{gray}{$\downarrow$} \\\hline
Avro none     & 237 &           76 &          113 &          5.4 &          168 &          8.3 &          3.7 \\
Avro snappy   & 198 &           79 &          116 &          6.2 &          172 &          8.8 &          4.3 \\
Avro deflate  & 180 &          142 &          117 &          7.6 &          197 &         13.5 &          8.9 \\
Avro LZMA     & 169 &          285 &          134 &         25.5 &          576 &        \textcolor{gray}{can't} &        \textcolor{gray}{can't} \\\hline
Parquet none  & 210 &        \textcolor{gray}{can't} &        \textcolor{gray}{can't} &        \textcolor{gray}{can't} &          177 &           54 &          2.3 \\
Parquet snappy  & 200 & \textcolor{gray}{$\downarrow$} & \textcolor{gray}{$\downarrow$} & \textcolor{gray}{$\downarrow$} &          184 &           54 &          2.4 \\
Parquet gzip  & 176 & \textcolor{gray}{$\downarrow$} & \textcolor{gray}{$\downarrow$} & \textcolor{gray}{$\downarrow$} &          199 &           56 &          2.5 \\
\end{tabular}




\end{document}
